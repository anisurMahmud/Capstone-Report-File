\chapter{Conclusions}\label{ch:conclusion}

\section{Conclusions}
In conclusion, Emotion Detection using LSTM model needed several intermediate layers for passing memory to get the desired output. In this project eight intermediate
layers are used to make the sequential model and to predict the emotion in 6 different categories [‘joy’, ‘sadness’, ‘fear’, ‘anger’, ‘love’, ‘surprise’]. Dataset of 16000 records has been used. In this sequential model, we used GloVe (Global Vectors for Word Representation) an unsupervised learning algorithm for creating word embeddings, which are dense vector representations of words in a continuous vector space and also used the genism library to load the GloVe pre-trained word embeddings model with 300 dimensions.

\section{Future Prospects of Our Work}
In future work, we plan to extract the more effective features of emotion and enhance the emotion detection accuracy. In Emotion detection, majority of records belong to either joy and sadness, therefore more number of records should be added for more accurate and diverse output. Also till now it is predicting the emotions in six different categories, it must be improved to categories in at least 15 emotions. Also use of some google API for detecting the emotion in any language can also be implemented. Performance can be improved with deep learning model via applying different word embedding techniques and various hyper parameter and also with several million words.
