\documentclass[12pt,notitlepage,oneside]{report}
\setcounter{secnumdepth}{3}
\usepackage{buet_msc_thesis}
\usepackage{lipsum}
\usepackage{sectsty}
\chapternumberfont{\fontsize{17}{15}\selectfont}
\chaptertitlefont{\fontsize{17}{15}\selectfont}
\sectionfont{\fontsize{14}{15}\selectfont}
\subsectionfont{\fontsize{12}{15}\selectfont}
%\usepackage[acronym, toc]{glossaries}

% Uncomment any of the following lines should you need to
% suppress the LOF, or LOT or LOA

% \suppresslistoffigures 
% \suppresslistoftables
% \suppresslistofalgorithms

% For index creation, comment this out if you do not want to create an
% index
\makeindex[intoc]
\patchcmd{\titlepage}{empty}{fancy}{}{}
\patchcmd{\chapter}{plain}{fancy}{}{}
\begin{document}


% Edit as needed below this line
% %%%%%%%%%%%%%%%%%%%%%%%%%%%%%%%%%%%%%%%%%%%%%%%%%%%%%%%%
% Chapter-1

\chapter{Introduction}\label{intro}
 At present there are millions of sites, applications, blogs that deal with text data. Most of the data deals with complex emotions of the people. Sometimes it is about a bad product, sometimes it is simply about someone's mental health condition. In order to sort all the data and make sense of it, a system is necessary.And it is not possible for a human being to make sense of millions and billions of data. So different language processing methods was created. Methods that can give results after processing thousands of data. These methods include models like RNN, LSTM, NLP.

Long short term memory or for short, LSTM is type of recurrent neural network. It is more like a modified version which can easily remember past data in memory. LSTM uses back propagation to train itself. So the solution we discuss is to implement LSTM to detect emotions more accurately from texts using better computation methods. 

\section{Problem Statement}
The problem is that we have huge data and not enough ways to process it and get a intended outcome. Existing models don't always give accurate results on the emotion the textual data is expressing. It can easily be identified by a human but it needs to be done by machines or simply computers more efficiently using less resources while computing the result. Because a lot of the existing methods rely on heavy computational resources. By using LSTM and modifying to give better output. 

We implement emotion detection and train it with datasets from kaggle which contains data consisting of mainly four emotions. Deep learning methods has been utilised for sentence analysis and processing of emotional data. Various approaches exist today for detecting emotion from text but often it is only limited to three categories, positive, negative and neutral. Deep learning technique provides the ability to classify different emotions as well from text data.
\section{Problem Background}
Neural networks plays an important role in the world of deep learning. There are a lot of neural network models which are used for this purpose. But as discussed earlier, not all of them provides efficient results. For example, CNN can also be used to identify emotions from texts but it has a drawback, it doesn't store the trained knowledge for it to be used later. The network needs to be trained every time we want to run tests. It starts from scratch again. That's why older neural network models can't utilize the benefit of memory storage. That's why we have deep learning capable algorithms such as LSTM, BI-LSTM which learns each and every word from the given sentences present in the corpus.   

\section{Motivations}

We are currently seeing a boom in social media popularity and ecommerce sites. More than half of the world uses social media for different interactions and a quarter of the people now prefer ecommerce sites for purchases of different necessities. The use of texts in these sites is seen excessively as people share their opinions as posts, blogs and reviews in social media and reviews of products in different ecommerce sites. As a result, we now have a very big resource in our hand which can be used for providing tailor made services for the customers by service providers. Determining emotion from texts can help us reach a verdict on a topic or a product from thousands or millions of people. It will be possible to determine whether a service is appreciated or hated by the people simply from their posts, reviews in an instant, saving tons of time and playing a big role in product improvement. Furthermore, it can be also applied in applications such as emotion retrieval from suicide notes, detecting insulting sentences in chat conversations and so on. Therefore, a better method of detecting emotion from text based data is very crucial in the modern world.

\section{Research Objectives}
There are many emotion detection techniques but not all of them provide state of
the art accuracy. There are abundant ways to make it more efficient and less time
consuming on the hardware capability end. The objectives of this project include –
\begin{itemize}
\item To better understand the concept of LSTM model and its word embedding techniques
\item To improve accuracy of emotion detection from texts using LSTM model
\item To detect wide range of emotions from given text in terms of anger, fear, joy and sadness 
\item To build a better algorithm to be implemented in applications like sentiment analysis
\end{itemize}

\section{Significance of the Research} 
Emotion detection allows the ability to extract insights from comments on social media, survey data or other sources of feedback methods. These insights are often helpful in understanding targeted people's perspective. It can also be used to determine mental condition of suicidal patients from their diaries or blogs. Which is why emotion detection is greatly important in this digital era. Therefore accurate results from such research will be of great importance to the world. 

\section{Key Contribution}
This research offers its contribution in 
\begin{itemize}
\item Improving accuracy of LSTM by fine tuning its architecture 
\item Improving performance with data training 
\item Optimizing it to have better learning rate and proper weight initialization.
\end{itemize}

% Chapter-2
\chapter{Literature Review} \label{ch:literature_review}


\section{Literature Review}

\section{Problem Analysis}








% Chapter-3
\chapter{Proposed System}
\label{ch:Proposed System}
Emotion detection from text has now become one of the key feature to understand peoples emotion. Everyone writes about their feelings or emotion in their social media, comment box or in inbox. If a system can identify those emotions from texts then it will be very help full to understand the people's thoughts. For this reason we build a system that can detect emotion from text. In this chapter we represents our architectural diagram and detailed module design of our text based emotion detection system.

\section{Design}
Here we represent the architectural design of the system. In our system we use text data from different anonymous chat data. The chat data is in text format. Data already includes different emotions specified. Then this data are sent to the finale stage where the emotion is analysed using LSTM network architecture.


\section{Module Description}
Here we provide detailed information about the modules of our system. That defines how our system works.
\begin{itemize}
\item Data collection
\item Data Preprocessing
\item Glove Embedding Vector Model
\item LSTM
\end{itemize}

\subsection{Data Collection}
To run the system at first we need a data set from where we can perform emotion detection. For our system, dataset was collected from kaggle.com. It contains about 20000 anonymous chat data from different whatsapp groups. The entire dataset is divided into three parts for testing, training and validation. The train dataset contains 16000 text data and test and validation dataset contains 2000 text data each.

\subsection{Data Preprocessing}
Preprocessing involves removing garbage value or noise from dataset. There's possibility of irrelevant data presence in dataset which are not necessary and therefore discarding them would benefit the model's training capacity and overall accuracy in the end. In our textual dataset, preprocessing involved turning all upper case letters in lower case letters, removing stop signs like full stop, coma and question mark signs.   

\subsection{Glove Embedding Vector Model}
GloVe, short for Global Vectors for Word Representation, is an unsupervised learning algorithm for obtaining vector representations (embeddings) of words.GloVe constructs a global word-word co-occurrence matrix from large text corpora and then factorizes it to capture the semantic relationships between words. The resulting embeddings encode meaningful semantic relationships and are widely used in natural language processing tasks such as text classification and sentiment analysis. GloVe embeddings are pretrained on extensive datasets, making them effective for capturing nuanced language patterns and aiding in tasks where understanding the context and meaning of words is crucial~\cite{ref10}.

\subsection{LSTM}
LSTM stands for long short-term memory networks which is used in the field of Deep Learning.LSTM is a advanced version of recurrent neural networks (RNNs) that is design to learn long-term dependencies, especially in sequence prediction problems.

\subsubsection{Architecture of LSTM}
The role of an LSTM model is held by a memory cell known as a cell state that maintains its state over time. The cell state is the horizontal line that runs through the top of the below diagram. Information can be added to or removed from the cell state in LSTM and is regulated by gates. These gates optionally let the information flow in and out of the cell. It contains a point-wise multiplication operation and a sigmoid neural net layer that assist the mechanism. The sigmoid layer gives out numbers between zero and one, where zero means nothing should be let through, and one means everything should be let through.~\cite{ref9}


\begin{figure}[ht!]
  \includegraphics[width=\linewidth]{chapters/LSTM.png}
  \caption{LSTM architecture}
\end{figure}

There are mainly five layers in LSTM network. But layer an be added to improve the accuracy of the model. The five layers are:
\begin{itemize}
\item \textbf{Embedding Layer:} This Layer is responsible for converting word tokens into embedding of a specific size like 256, 512 etc. This layer maps a sequence of word indices to embedding vectors and learns the word embedding during training.
In this layer the word tokens are converted into embedding of a specific size like 256,512 etc. This layer also responsible for mapping a sequence of word indices to embedding vectors and learns the word embedding during training.

\item \textbf{LSTM Layer:}This layer sequentially process data and keep its hidden state through time. It keeps previous memory and process with current memory and decides whether to keep current memory or remain with old memory.
   
\item \textbf{Fully Connected Layer:} This layer indicates those layers where all inputs from one layer are connected to every unit of the next layer.

\item \textbf{Sigmoid Activation Layer:} This layer is responsible for converting all output values in the range of 0 to 1. It means it can either let no flow or complete flow of information throughout the gates.

\item \textbf{Output Layer:} Output layer is the final layer of the architecture. This layer is responsible for generating output which is obtained from sigmoid layer. The output formate is 2d array of real numbers where the first dimension indicates the number of samples given to the LSTM layer and second dimension is the dimensionality of the output space defined by the units
parameter in Keras LSTM implementation.
\end{itemize}


% Chapter-4
\chapter{Implementation} \label{ch:implementation}
Theoretically everything works on paper but a method can not be justified completely unless it is implemented and tested as well. This chapter emphasizes on requirements for running the model and how the model is actually built and its working principle.

\section{System Requirements}
Deep learning models often require heavy hardware to operate faster. These models can be run on lower end desktop computers but training the data and testing it would take a lot of time. There are two methods of working on these models
\begin{itemize}
\item Using PAAS Services namely Google Colab
\item Using Jupyter Notebook on local machine
\end{itemize} 
\subsection{Using PAAS Services}
Platform as a service or simply PAAS provides necessary hardware and software support for development purposes based on subscription or fees. For our model, Google Colab is a great service that provides great support to a certain range for free of cost. It is a hosted jupyter notebook service that requires no setup to use while providing resources to build models for example gpus~\cite{url1}. Only requirement in this case is a stable internet connection and a functioning computer device with a browser.
\subsection{Using Jupyter Notebook}
Jupyter notebook is a web-based interactive computational environment which is used by wide variety of developers for managing and integrating big data tools~\cite{url2}. In order to use it on local machine, one will need certain hardware capabilities to work with machine learning models.
\begin{table}
\caption{Hardware and OS Requirement for Jupyter Notebook}
\begin{tabular}{|l|l|}
\hline
Type & Specification \\
\hline
CPU & Any Quad core CPU with at least 2.1 GHZ base speed \\
\hline
Ram & 8 gigabytes \\
\hline
GPU(Optional) & 8 gigabytes of vram or higher for fast computation \\
\hline
OS & Linux, Windows, MacOS \\
\hline
\end{tabular}
\end{table}



% Chapter-5
\chapter{Testing and Result Analysis}




% Chapter-6
\chapter{Conclusions}\label{ch:conclusion}

\section{Conclusions}
In conclusion, Emotion Detection using LSTM model needed several intermediate layers for passing memory to get the desired output. In this project seven intermediate
layers are used to make the sequential model and to predict the emotion in 6 different categories [‘joy’, ‘sadness’, ‘fear’, ‘anger’, ‘love’, ‘surprise’]. Dataset of 16000 records has been used. In this sequential model, we used GloVe (Global Vectors for Word Representation) an unsupervised learning algorithm for creating word embeddings, which are dense vector representations of words in a continuous vector space and also used the genism library to load the GloVe pre-trained word embeddings model with 300 dimensions.

\section{Future Prospects of Our Work}
In future work, we plan to extract the more effective features of emotion and enhance the emotion detection accuracy. In Emotion detection, majority of records belong to either joy and sadness, therefore more number of records should be added for more accurate and diverse output. Also till now it is predicting the emotions in six different categories, it must be improved to categories in at least 15 emotions. Also use of some google API for detecting the emotion in any language can also be implemented. Performance can be improved with deep learning model via applying different word embedding techniques and various hyper parameter and also with several million words.



% Chapter showing example of index creation
%\input{indexcreation.tex}


% Bibliographies and appendices
	
\renewcommand{\bibname}{References}
\begin{thebibliography}{8}
\bibitem{ref1}
Wang X, Kou L, Sugumaran V, Luo X, Zhang H. Emotion correlation mining through deep learning models on natural language text. IEEE transactions on cybernetics. 2020 May 13;51(9):4400-13.

\bibitem{ref2}
Fei H, Ji D, Zhang Y, Ren Y. Topic-enhanced capsule network for multi-label emotion classification. IEEE/ACM Transactions on Audio, Speech, and Language Processing. 2020 Jun 10;28:1839-48.

\bibitem{ref3}
Krommyda M, Rigos A, Bouklas K, Amditis A. An experimental analysis of data annotation methodologies for emotion detection in short text posted on social media. InInformatics 2021 Mar 12 (Vol. 8, No. 1, p. 19). MDPI.

\bibitem{ref4}
Chiorrini A, Diamantini C, Mircoli A, Potena D. Emotion and sentiment analysis of tweets using BERT. InEDBT/ICDT Workshops 2021 Mar 23 (Vol. 3).

\bibitem{ref5}
Pepino L, Riera P, Ferrer L, Gravano A. Fusion approaches for emotion recognition from speech using acoustic and text-based features. InICASSP 2020-2020 IEEE International Conference on Acoustics, Speech and Signal Processing (ICASSP) 2020 May 4 (pp. 6484-6488). IEEE.

\bibitem{ref6}
Yang X, Feng S, Zhang Y, Wang D. Multimodal sentiment detection based on multi-channel graph neural networks. InProceedings of the 59th Annual Meeting of the Association for Computational Linguistics and the 11th International Joint Conference on Natural Language Processing (Volume 1: Long Papers) 2021 Aug (pp. 328-339).

\bibitem{ref7}
Graterol W, Diaz-Amado J, Cardinale Y, Dongo I, Lopes-Silva E, Santos-Libarino C. Emotion detection for social robots based on NLP transformers and an emotion ontology. Sensors. 2021 Feb 13;21(4):1322.

\bibitem{ref8}
Sharma T, Diwakar M, Singh P, Lamba S, Kumar P, Joshi K. Emotion Analysis for predicting the emotion labels using Machine Learning approaches. In2021 IEEE 8th Uttar Pradesh Section International Conference on Electrical, Electronics and Computer Engineering (UPCON) 2021 Nov 11 (pp. 1-6). IEEE.

\bibitem{ref9}
Yu Y, Si X, Hu C, Zhang J. A review of recurrent neural networks: LSTM cells and network architectures. Neural computation. 2019 Jul 1;31(7):1235-70.

\bibitem{ref10}
Mohammed SM, Jacksi K, Zeebaree SR. Glove word embedding and DBSCAN algorithms for semantic document clustering. In2020 International Conference on Advanced Science and Engineering (ICOASE) 2020 Dec 23 (pp. 1-6). IEEE.

\bibitem{url1}
Colab, \url{https://colab.google/}. Last Accessed December 1, 2023

\bibitem{url2}
Project Jupyter, \url{https://en.wikipedia.org/wiki/Project_Jupyter}. Last Accessed December 1, 2023

\end{thebibliography}

% Index, comment this out if you do not want to create an index
\printindex

\appendix
% Algorithms
%\input{chapters/algorithms.tex}

\end{document}